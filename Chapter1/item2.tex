\section{条款2:理解auto类型推导}

如果你已经阅读了条款1,那么你几乎已经掌握了关于auto类型推导的全部知识,因为除了一个例外之外,auto类型推导几乎就是模板类型推导。但是怎么会呢?模板类型推导包括了模板、函数和形参,但是auto并不处理它们中的任一个。

事实确实如此,但是也并没有关系。模板类型推导和auto类型推导之间存在一个映射。通过一种逐字逐句的算法进行互相转换。

在条款1中,模板类型推导使用了如下的函数模板:

\begin{lstlisting}
template<typename T>
void f(ParamType param);
\end{lstlisting}

和如下的函数调用:

\begin{lstlisting}
f(expr); //使用表达式调用f
\end{lstlisting}

在f的函数调用中,编译器使用expr去推导T和ParamType的类型。当一个变量使用auto声明时,auto扮演了模板中T的角色,变量的类型说明符(specifier)则扮演了ParamType的角色。这里用以下的例子能够更好的描述这个情形:

\begin{lstlisting}
auto x = 27;
\end{lstlisting}

这里x的类型说明符就是auto本身。从另一方面来讲,在这个声明中,

\begin{lstlisting}
const auto cx = x;
\end{lstlisting}

的类型说明符是const auto。并且在这里,

\begin{lstlisting}
const auto& rx = x;
\end{lstlisting}

的类型描述符是const auto\&。为了推导例子中x,cx和rx的类型,编译器会认为每个声明都是一个模板,并且按照模板的方式来初始化表达式:

\begin{lstlisting}
template<typename T>
void func_for_x(T param); //推导x类型的概念模板(conceptual template)

func_for_x(27); //概念调用:param的推导类型是x的类型

template<typename T>
void func_for_x(const T param); //推导cx类型的概念模板

func_for_cx(x); //概念调用:param的推导类型是cx的类型

template<typename T>
void func_for_rx(const T& param); //推导rx类型的概念模板

func_for_rx(x); //概念调用:param的推导类型是rx的类型
\end{lstlisting}

如我所说,除了一个例外外,auto的类型推导与模板的类型推导规则一样。

条款1基于ParamType的特点、函数模板中param的类型描述符,将模板类型的推导分为3种情形。在使用auto的变量推导中,类型描述符取代了ParamType的位置,因此这里同样也有3种情形:

\begin{itemize}
\item{情形1:类型描述符是一个指针或引用,但不是一个通用引用。}
\item{情形2:类型描述符是一个通用引用。}
\item{情形3:类型描述符既不是一个指针,也不是一个引用。} 
\end{itemize}

我们已经看到了情形1和情形3:

\begin{lstlisting}
auto x = 27; //情形3,x不是指针,也不是引用。

const auto cx = x; //情形3,cx也不是。

const auto& rx = x; //情形1,rx是一个非通用引用。
\end{lstlisting}

情形2与你期望的一样:

\begin{lstlisting}
auto&& uref1 = x; //x是一个int类型的左值表达式,所以uref1是int&

auto&& uref2 = cx; //cx是一个const int类型的左值表达式,所以uref2是const int &

auto&& uref3 = 27; //27是一个int类型的右值表达式,所以uref3是int&&
\end{lstlisting}

条款1总结了数组和函数如何退化为非引用类型指针的类型描述符。这种情形也在auto类型推断中发生了:

\begin{lstlisting}
const char name[] = "R. N. Briggs"; //name的类型是const char[13]

auto arr1 = name; //arr1的类型是const char *

auto& arr2 = name; //arr2的类型是const char &[13]

void someFunc(int, double); //someFunc是一个void(int, double)类型的函数

auto func1 = someFunc; //func1的类型是void(*)(int, double)

auto& func2 = someFunc; //func2的类型是void(&)(int, double)
\end{lstlisting}

如你所见,auto类型推导与模板类型推导非常相似。它们就像是硬币的两边。

除了一个例外。我们将会通过一个例子开始:你希望声明一个初值为27的int对象,C++98给了2种方法:

\begin{lstlisting}
int x1 = 27;
int x2(27);
\end{lstlisting}

通过C++11中的同意初始化(Uniform Initialzation),加入了这些方法:

\begin{lstlisting}
int x3 = {27};
int x4{27};
\end{lstlisting}

总共使用了4种语法,获得了同样的结果:一个初值为27的int对象。

但如条款5中描述的那样,使用auto类型相较于使用固定类型声明变量有许多好处。因此使用auto替换上述例子中的int将会非常愉快的。简单的文本替换后,变成了如下代码:

\begin{lstlisting}
auto x1 = 27;
auto x2(27);
auto x3 = { 27 };
auto x4{ 27 };
\end{lstlisting}

这些声明都能够通过编译,但是它们却并不代表相同的含义。前两个声明确实使用27声明了1个int类型的变量。然而后两个却是声明了一个拥有一个元素27的std::initializer\_list<int>类型的变量!

\begin{lstlisting}
auto x1 = 27; //类型是int,值是27
auto x2(27); //同上
auto x3 = { 27 }; //类型是std::initializer_list<int>,值是{27}
auto x4{ 27 }; //同上
\end{lstlisting}

这种情形是源于auto的特殊类型推导规则。当一个auto声明的变量的初始化器(Initializer)放置在一对大括号中时,这个变量的推导类型是的std::initializer\_list。如果这个类型不能被推导(例如大括号中的值都是不同类型的),这段代码将会无法通过编译。

\begin{lstlisting}
auto x5 = {1, 2, 3.0}; //错误!不能为std::initializer_list<T>推导T
\end{lstlisting}

如注释所指出的那样,在这种情形中,类型推导失败了。但是理解在这个地方实际上发生了两种类型的类型推导是很重要的。一种源于auto的使用:x5的类型必须被推导。因为x5的初始化器在大括号中,所以x5必须被推断为std::initializer\_list。第二种则是,因为std::initializer\_list是一个模板,std::initializer\_list<T>的为某种类型的T实例化,这也意味着T的类型必须被推导。上述的类型推导是因为第二种:模板类型推导而失败的,因为大括号中的初始化器中的元素有多种类型。

对于大括号初始化器的不同的处理方式是auto类型推导和模板类型推导唯一不同的地方。当auto声明的变量被大括号初始化器初始化时,推导出的类型是std::initializer\_list。但是如果相应的模板被传入一个相同的初始化器时,类型推导会失败,代码无法通过编译。

\begin{lstlisting}
auto x = {11, 23, 9} //x的类型是std::initializer_list<int>

template<typename T>
void f(T param); //与x声明等价的模板形参声明

f({11, 23, 9}); //错误,无法为T推导类型
\end{lstlisting}

然而,当你指定形参的类型是std::initializer\_list<T>时,模板类型推导规则就会成功推导出T的类型:

\begin{lstlisting}
template<typename T>
void f(std::initializer_list<T> initList);

f({11, 23, 9}); 
//T被推导为int,initList的类型是std::initializer_list<int>
\end{lstlisting}

所以auto和模板类型推导的唯一差别就是:auto假定大括号初始化器代表着std::initializer\_list,而模板类型则不同。

你可能希望知道为什么auto类型推导对于大括号初始化器使用了特别的规则,但是模板类型推导没有。我也想知道,但是不幸的是,我没有找到一个方便的解释。因为规则就是规则,这意味着你在使用auto声明一个变量,并使用一个大括号初始化器时必须记住:推导出的类型一定是std::initializer\_list。如果你想更深入的使用统一的集合初始化时,你就更要牢记这一点。C++11中一个最经典的错误就是程序员意外的声明了一个std::initializer\_list类型的变量,但他们的本意却是想声明一个其他类型的变量。错误造成的主要原因是一些程序员只有当必要的时候,才使用大括号初始化器进行初始化。(将会在条款7中详细讨论。)

对于C++11来说,这已经是一个完整的故事了,但是对于C++14,故事还没有结束,C++14允许auto来表示一个函数的返回值的类型(见条款3),并且C++14的lambda表达式可以在参数的声明时使用auto。不管怎样,这些auto的使用,采用的都是模板类型推导的规则,而不是auto类型推导规则,这意味着,大括号的初始化式会造成类型推导的失败,所以一个带有auto返回类型的函数如果返回一个大括号的初始化式将不会通过编译。

\begin{lstlisting}
auto createInitList() 
{
	return {1, 2, 3}; //错误,不能够推导类型f{1, 2, 3}
}
\end{lstlisting}

同样,规则也适用于当auto用于C++14的lambda的参数类型说明符时:

\begin{lstlisting}
std::vector<int> v;
...

auto resetV = 
	[&V](const auto& newValue) { v = newValue; }; //C++14
...

resetV( {1, 2, 3} ); //错误,不能够推导类型f{1, 2, 3}
\end{lstlisting}

\begin{mdframed}
请记住:
\begin{itemize}
\item{auto类型推导规则通常与模板类型推导相同,但是auto类型推导假定1个大括号初始化器代表着std::initializer\_list,而模板类型推导不然。}
\item{当auto是一个函数的返回值类型或是一个lambda传递形参类型时,使用模板类型推导规则,而不是auto类型推导规则。} 
\end{itemize}
\end{mdframed}
